% Activate the following line by filling in the right side. If for example the name of the root file is Main.tex, write
% "...root = Main.tex" if the chapter file is in the same directory, and "...root = ../Main.tex" if the chapter is in a subdirectory.
 
%!TEX root =  ../main.tex

\chapter[Transforms and distances]{Common data transforms in high throughput sequencing}

Fundamentally, the goal of any experiment is to determine something about the environment that was sampled. After all, we are attempting to use HTS to determine something of interest about the underlying environment. Thus, we need to have some equivalence between the samples before sequencing and the samples after sequencing. The simplest case would be that there would be a linear relationship between the data that we could obtain from the environment, and the data that was actually collected by HTS. 

We can think about the underlying data on a univariate basis; do the features across all samples follow a Gaussian distribution? or do they follow some unknown distribution? If so, can we transform the data to approximate a Gaussian distribution? This mode of thinking leads to the use of square-root, arcsine or Hellinger transformations since they appear to transform the data into a distribution that can be interpreted. However, as we shall see below, none of these univariate transformations  is suitable. 

It is more desirable to think about HTS data in a multivariate way as a `composition'  because the total count of molecules in the underlying sample (the environment) is always a confounding variable \footcite{Loven:2012aa}. This way of thinking led to multivariate data normalizations. 

\section{Simple proportional type transformations}

We use the following notation throughout. Column vectors contain samples $\vec{\textbf{s}}$ and row vectors contain features $\vec{\textbf{f}}$.  There are $D$ features and $n$ samples, thus the data are contained in matrix $M = D \times n$. The $j^{th}$ sample is denoted as $s_{j}$, the $i^{th}$ feature of all samples is denoted as $s_{i-}$, and the value for the $i^{th}$ feature of the $j^{th}$ sample is referred to as $s_{ij}$.

The simplest normalization is to determine the relative abundance (rAB), or proportion, of the \ith feature in a sample as in Eq. \ref{eq:rab}.

\marginpar{
\begin{equation}
	rAB_{i} = \frac{s_{i}}{\sum{\vec{\textbf{s}}}}
	\label{eq:rab}
\end{equation}
}

This normalization is also referred to as the total sum scaling (TSS) normalization. The rAB measure requires only the read count observed for a the feature $s_i$ and the total read count of the sample $\sum{\vec{\textbf{s}}}$. Since this measure is generally skewed, it is often log-transformed prior to analysis. 

A further normalization was proposed early in the RNA-seq field where the reads per kilobase per million mapped (RPKM)\footcite{Mortazavi:2008} method was used initially to place the read counts for each feature within and between samples on a common scale. 

For this we also needed to know a scaling factor $K$, and the length of the feature $L_i$; from this, the RPKM value for the \ith feature for each sample was calculated as in Eq. \ref{eq:rpkm}.

\marginpar{
\begin{equation}
	RPKM_{i} = \frac{(K \cdot C_{i} )}{\sum{C} \cdot L_{i}}
	\label{eq:rpkm}
\end{equation}
}
When the equation is placed in this form it is obvious that RPKM is simply a scaled rAB where each rAB value is divided by its length multiplied by a constant. In compositional terms, RPKM is an unclosed perturbation of the original data.

Further research suggested that RPKM was not appropriate for comparison of features between samples. The goal of RPKM was to `count' reads per feature per cell.  In the original paper the authors supplied an equivalence and an RPKM value of 1 RPKM equalled one transcript in each cell in the C2C12 cell line, but in liver cells, a value of 3 RPKM equalled one transcript per cell. Thus, from the start, this normalization was unable to normalize between-condition read counts.  

The transcripts per million (TPM) normalization was advocated next \footcite{Li:2010aa}. Patcher \footcite{Pachter:2011} showed the equivalence between RPKM and TPM, and in compositional terms TPM is simply a compositionally closed form of RPKM multiple by a constant as in Eq. \ref{eq:tpm}.

\marginpar{
\begin{equation}
	TPM_{i} = \frac{RPKM_i}{\sum{RPKM}} \cdot K
	\label{eq:tpm}
\end{equation}
}

The rAB, RPKM and TPM normalizations are thus all very similar, differing only in the scaling of individual features, and do not allow normalization between conditions unless the conditions contain the same input number of RNA molecules. In a very real sense, these normalizations deliver proportional data, scaled or perturbed to make the data appear as if they are numerical, and not proportional.

A related transformation is `rarefaction' or subsampling without replacement to a defined per-sample read count. This transformation was widely used in the 16S rRNA gene sequencing field. Rarefaction to a common read count gives a composition, that is scaled such that low count features often are replaced by 0 values \footcite{McMurdie:2014a}. For this reason, rarefaction has now been largely replaced with the median of ratios method described below. 

\section{The median of ratios count normalization} 

Further work found that none of these methods were appropriate, since the read count per sample continued to confound the analyses \footcite{Loven:2012aa}. Thus, the scaling normalization methods were proposed \footcite{Robinson:2010a}. There are two main scaling normalizations, but both operate on the common assumption that by normalizing all counts in a sample to a midpoint of each sample that the normalization can impute the {\em number} of each feature in the environment.  The approaches differ largely in how the midpoint is determined. The median of ratios method (MR) is instantiated in DESeq2 (and others), and the trimmed mean of M values (TMM) method is used by edgeR (and others). The DM method will be demonstrated and used, but the TMM gives substantially similar results, and uses the same basic logic since values are scaled by a feature-wise midpoint.

The DM method calculates the ratio of the features to the geometric mean, $\mathrm{G}_i$, of each feature across all samples, and then takes as the normalization factor the median ratio per sample as the scaling factor. Each feature is then divided by the scaling factor to place each sample on an equivalent count scale. The idea is that the DM normalization `opens' the data from being compositional to being scaled counts. As we shall see, it is impossible to open the data, and while the scaled counts have some useful properties, removing  compositional constraints are not among them. 

The multi-step normalization MR normalization attempts to normalize for sequencing depth thus `opening' the data, and proceeeds as in the multistep Eq. \ref{eq:dm}. Here we start with two sample vectors $\vec{\textbf{s}}_1$ and $\vec{\textbf{s}}_2$, and calculate a vector of geometric means of the features $\vec{\textbf{g}}$. Ratio vectors, $\vec{\textbf{r}}_j$ are calculated by dividing the sample vectors by the geometric mean vector, and the median of the ratio vectors is determined. Finally, the sample vectors are divided by the median of the ratio vector for each sample. 


\marginpar{
\begin{equation}
	\begin{aligned}
		\vec{\textbf{g}} = &\ \mathrm{G}_{i-}\\
		\vec{\textbf{r}}_j = &\ \vec{\textbf{s}}_j / \vec{\textbf{g}}\\
		\vec{\textbf{d}}_j = &\ \vec{\textbf{s}}_j / Md(\vec{\textbf{r}}_j)\\
	\end{aligned}
\label{eq:dm}
\end{equation}
}

In Table \ref{tab:des} we can see that the median ratio for each sample $\vec{\textbf{r}}_j$  samples may be different in each sample, and that the particular feature that is the median may itself be different, the median feature is in boldface in the table. Thus, by construction the feature values in each sample can be scaled by different amounts in each sample. 


\begin{table}[!h]
\caption{Example calculation of DM normalization}
\centering
\begin{tabular}{r r r r r r r r}
\hline
Feature & $\vec{\textbf{s}}_1$ & $\vec{\textbf{s}}_2$ & $\vec{\textbf{g}}$ & $\vec{\textbf{r}}_1$ & $\vec{\textbf{r}}_2$ & $\vec{\textbf{d}}_1$ & $\vec{\textbf{d}}_2$ \\ \hline \hline
F1 & 1500 & 1000 & 1224.7 & 1.22 & {\bf 0.81} & 1219.5 & 1234.6\\
F2 & 25 & 15 & 19.4 & 1.29 & 0.77 & 20.3 & 18.5 \\
F3 & 1000 & 500 & 707.1 & 1.41 & 0.71 & 813.0 & 617.3 \\
F4 & 75 & 50 & 61.2 & {\bf 1.23} &  0.82 & 61.0 & 61.7 \\
F5 & 500 & 1500 & 866.0 & 0.58 & 1.73 & 406.5 & 1851.9\\ \hline
\end{tabular}
\label{tab:des}
\end{table}

\section{Log-ratio transformations}
There are three main log-ratio transformations; the additive log-ratio (alr), centred log-ratio (clr) and the isometric log-ratio (ilr) \footcite{pawlowsky2015modeling}.

Using the same notation as above for a sample vector  $\vec{\textbf{s}}$ of $D$ `counted' features (taxa, operational taxonomic units or OTUs, genes, etc.) $\vec{\textbf{s}}=[s_1, s_2, ... s_D]$:

The alr is the simply the elements of the sample vector divided by a presumed invariant feature, which by convention here is the last one:

\marginpar{
\begin{equation}
\begin{aligned}
 \vec{\textbf{x}}_{alr}= &\ [log(x_1/x_D), log(x_2/x_D), \\
 & \ldots log(x_D-1/x_D]
\end{aligned}
 \label{eq:alr}
\end{equation}
}

This is similar to the concept used in quantitative PCR, where the relative abundance of the feature of interest is divided by the relative abundance of a (presumed) constant `housekeeping' feature. Of course there are two major drawbacks. First, that the experimentalist's knowledge of which, if any, features are invariant is necessarily incomplete. Second, is that the choice of the (presumed) invariant feature has a large effect on the result if the presumed invariant feature is not invariant, or if it is correlated with any other features in the dataset. Interestingly, an early proposal was to use the geometric mean of a number of internal controls \footcite{Vandesompele:2002aa}, leading to the next transformation.

The centered log-ratio (clr) transformation introduced by \footcite{Ait1983},\footcite{Aitchison:1986} uses the geometric mean of all features as the denominator:

\marginpar{
\begin{equation}
\begin{aligned}
   \vec{\textbf{x}}_{clr} = & [log(x_1/\mathrm{G}(\vec{\textbf{x}})), \\
   & log(x_2/\mathrm{G}(\vec{\textbf{x}})), \\
   & \ldots log(x_D/\mathrm{G}(\vec{\textbf{x}}))]
\end{aligned}
\label{eq:clr}
\end{equation}
}
where $\mathrm{G}(\vec{\textbf{x}}) = \sqrt[D]{x_1 \cdot x_2 \cdot ... \cdot x_D}$, the geometric mean of $\vec{\textbf{x}}$.

The clr is often criticized since it has the property that the sum of the clr vector must equal 0. This constraint causes a singular covariance matrix; i.e., the sum of the covariance matrix is always a constant \footcite{pawlowsky2015modeling}. However the clr has the advantage of being readily interpretable, a value in the vector is its abundance \emph{relative} to a mean value.

The ilr is the final transformation, and is a series of sequential log-ratios between two groups of features. For example, the philr transformation is the series of ratios between OTUs partitioned along the phylogenetic tree \footcite{Silverman:2017aa}, although any other sequential binary partitioning  scheme is also possible \footcite{pawlowsky2015modeling}. The ilr transformation does not suffer the drawbacks of either the alr or clr, but does not allow for insights into relationships between single features in the dataset.  Nevertheless, ilr transformations permit the full-range of multivariate tools to be used, and are recommended whenever possible.

The ilr and clr are directly comparable in a two important ways: First, the distances between samples computed using an ilr and clr transformation are equivalent. Second, the clr approaches the ilr in other respects as the number of features becomes large. In this respect, the large number of features---hundreds in the case of OTUs, thousands in the case of genes---in a typical experiment works in our favour. Thus, while not perfect, the clr is the most widely used transformation. However, care must be taken when interpreting its outputs since single features must always be interpreted as a ratio between the feature and the denominator used for the clr transformation. The problems of using clr are apparent  when some subcomposition or group of taxa is analysed for further insight since the geometric mean of the subcomposition is not necessarily equal to that of the original composition, leading to potential inconsistencies.

Log-ratio values of any type do not need to be normalized since the total sum is a term in both the numerator and the denominator. Thus, the same log-ratio value will be obtained for the vector of raw read counts, or the vector of normalized read counts, or the vector of proportions calculated from the counts. Thus, log-ratios are said to be equivalence classes such that there is no information in the total count (aside from precision) \footcite{barcelo:2001}. 

Attempts to `open' the data are doomed to failure because the data cannot be moved from the simplex to Euclidian space. The total count delivered by the sequencing instrument is a function of the instrument and not the number of molecules sampled from the environment, thus the total count has no geometric meaning. If the data are collected in such a way that the total count represents the actual count in the environment, then the data are not compositional and issues regarding compositional data disappear. However, at present all sequencing platforms deliver a fixed-sum, random sample of the proportion of molecules in the environment.

Note that this does not mean that the read depth is irrelevant since more reads for a sample translate into greater precision when estimating the proportions \footcite{fernandes:2013}.

\section{Distance metrics}

The microbiome and transcriptome literature are replete with distance metrics, and it is common to find that a single study will use several distance metrics to report their findings. This is a problem since it shows that practitioners are unsure of the reason to use a metric, and the use of more than one metric leads to data dredging and research degrees of freedom---both of which increase the chances of finding false positives in the data to a surety.

Distance metrics can be broadly divided into those that require partitioning and those that do not. The UniFrac \cite{Lozupone:2011aa,unifrac:2005} and philr \cite{Silverman:2017aa} both require a phylogenetic tree, making these metrics applicable only to situations where the features can be so partitioned. For example, these distances are useful when examining 16S rRNA gene sequencing experiments. We have found that the unweighted UniFrac method is unreliable, and should be used with caution \cite{Wong:2016aa}, a point that was made in the original UniFrac paper and subsequently forgotten. The philr metric is a drop-in replacement for the weighted UniFrac distance metric and should be used whenever possible, since `philr` is an ilr transformation of the data where the sequential binary partitions are made along the phylogenetic tree. The `philr` transformation is thus compositionally appropriate. In practice, the weighted UniFrac distance metric provides similar results to the Aitchison distance, described below, and the ilr distance calculated using the philr transform approaches the Aitchison distance when the number of features is large. 

Several non-phylogenetic distances are in widespread use in the literature. These will be discussed in turn below, and their effects on distances between a random samples illustrated.

\subsection{Distances in counts}
\subsection{Euclidian Distance of DM transformation}
\subsection{Bray-Curtis Dissimilarity}
\subsection{Jensen-Shannon Divergence}
\subsection{Aitchison Distance}
\section{Transformations and distance} 



